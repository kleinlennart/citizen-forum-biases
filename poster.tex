% Based on Gemini theme:
% https://github.com/anishathalye/gemini

% https://github.com/deselaers/latex-beamerposter
\documentclass[final]{beamer}

\usepackage[T1]{fontenc}
\usepackage{lmodern}
\usepackage[orientation=landscape,size=a0,scale=1.4]{beamerposter}
% all font scaled to 1.4 

\usetheme{gemini}
\usecolortheme{Tuebingen} % my custom Uni Tübingen Theme

\usepackage{graphicx}
\usepackage{booktabs}
\usepackage{pgfplots}
\pgfplotsset{compat=1.14}
\usepackage{anyfontsize}
\usepackage{physics} % bold math

\setlength{\tabcolsep}{10pt} % inter-column space (default: 6pt)

% Bibliography
\usepackage{natbib}
\bibliographystyle{chicago-apsr}
\setcitestyle{aysep={}}

% Lengths ====================

% If you have N columns, choose \sepwidth and \colwidth such that
% (N+1)*\sepwidth + N*\colwidth = \paperwidth
\newlength{\sepwidth}
\newlength{\colwidth}
\setlength{\sepwidth}{0.025\paperwidth}
\setlength{\colwidth}{0.3\paperwidth}

\newcommand{\separatorcolumn}{\begin{column}{\sepwidth}\end{column}}

% Header ====================

\title{Should I Go? Participation Biases in  Citizen\\Deliberation Forums: Evidence from Germany}
\author{Lennart Klein \inst{1}}
\institute[shortinst]{\inst{1} University of Tübingen}

% Footer ====================

\footercontent{
  \href{https://github.com/kleinlennart/citizen-forum-biases}{https://github.com/kleinlennart/citizen-forum-biases} \hfill
    46th Annual Meeting of the International Society of Political Psychology\hfill
  \href{mailto:lennart.klein@student.uni-tuebingen.de}{lennart.klein@student.uni-tuebingen.de}}


% Logo ====================

\logoright{\includegraphics[height=5cm]{images/logo_uni_white.pdf}}

% Body ====================

\begin{document}

\begin{frame}[t]
\begin{columns}[t]
\separatorcolumn

% ----------------------------------------------------------------
\begin{column}{\colwidth}

  \begin{block}{Motivation}

  \begin{itemize}
        \item Increasing popularity of Citizen's Councils in Germany (\textbf{Table \ref{tab:councils}})
        \item Climate activists group \textit{``Letzte Generation''} demands a representative Standing Citizen's Council on Climate Change with legislative power
        \item New Problems, New Institutions? (``Climate Democracy'')
  \end{itemize}

  \vspace{1ex}
  > \textbf{``How attitudinally and psychologically representative and inclusive are Citizen's Councils?''}        
  \vspace{2ex}


  
  \end{block}

  \begin{block}{Research Background}

    \heading{Deliberative Democracy}

    \begin{itemize}
        \item James S. Fishkin's \citeyearpar{fishkin1991DemocracyDeliberation} classic work on \textit{``Deliberative Polling''}
    \end{itemize}
    
    \textit{\textbf{Important Characteristics:}}
        \begin{enumerate}
            \item Random selection
            \item Voluntary participation
            \item Political equality
            \item Representativeness
            \item Inclusivity
        \end{enumerate}

\textit{\textbf{The Problem:}}

    \begin{itemize}
        \item Usually very \textbf{high non-response rates} \citep[e.g.,][]{dean2022CitizenDeliberation}
        \item Potentially \textbf{high self-selection bias}
    \end{itemize}

    \heading{Participation Biases}

    \begin{itemize}
        \item \citet{luskin2022DeliberativeDistortions} have explored the social biases during the deliberation process
        \item  But what about the political and psychological participation biases through self-selection?
    \end{itemize}

  \end{block}
\end{column}
\separatorcolumn


% Column 2 ––––––––––––––––––––––––––––––––––––––––––––––––––––––––––––––––––––––––––––––––––––––––

\begin{column}{\colwidth}

  \begin{exampleblock}{Hypotheses}

    \heading{H1: \textit{Attitudes} – Participants report higher levels of political\\ interest and trust in institutions than the general public.}
    \heading{H2: \textit{Personality} – Participants tend to score higher on\\ Assertiveness and Extraversion than the general public.}
    
  \end{exampleblock}


  \begin{block}{Methods}

    \heading{Sampling \& Data Collection}
    \vspace{-1ex}
    \begin{itemize}
        \item Online survey of current and former participants of multiple national and regional citizen's councils
        \item Survey items matched with constructs in the \textit{German General Social Survey 2021} \textit{(ALLBUS)} 
        \item Big-Five Factor Scale \citep{goldberg1992DevelopmentMarkers} compared to national dataset from \textit{Open Psychometrics}
    \end{itemize}
  \end{block}

    \begin{block}{A new ‘Deliberative Wave’?}

    \begin{itemize}
        \item Citizen's Council participants: $M = \vb{186.88}$ ($SD = 134.19$)
        \item Example: Respondents in the monthly representative \textit{DeutschlandTREND} survey: $N = \vb{1.000 - 1.500}$
    \end{itemize}


\begin{table}[]
\label{tab:councils}
\caption{Size of German National Citizen's Councils}
\begin{tabular}{@{}lll@{}}
\toprule
\textbf{Name} & \textbf{Year} & \textbf{N} \\ 
\midrule
Citizens' Council on Democracy & 2019 & 160 \\
Citizens' Council \textit{``Germany's Role in the World''} & 2020 & 160 \\
Citizens' Council on Education and Learning & 2021 & 500 \\
Citizens' Council on Climate Change & 2021 & 160 \\
Citizens' Council on Artificial Intelligence & 2022 & 200/100* \\
Citizens' Council on Research & 2022 & 55 \\
Citizens' Forum on the Future of Europe & 2022 & 100 \\
Citizens' Council \textit{``Change in Nutrition''} & 2023 & 160 \\
\bottomrule
\small\textit{Notes:} Data from \textit{Mehr Demokratie e.V.},  *with control group
\end{tabular}
\end{table}

\end{block}
  
\end{column}
\separatorcolumn

%% Column 3 ––––––––––––––––––––––––––––––––––––––––––––––––––––––––––––––––––––––––––––
\begin{column}{\colwidth}

  \begin{alertblock}{Implications \& Discussion}
    \heading{Normative Democratic Theory and Practice}
    \vspace{-0.5ex}
    \begin{itemize}
        \item Citizen's Councils might not be able to deliver on their ``democratic ideal'' of true representativeness
        \vspace{0.5ex}
        \item Could potentially even backfire by alienating less engaged members of the general public, increasing political apathy
    \end{itemize}

    \heading{Institutional Design}
    \vspace{-0.5ex}
    \begin{itemize}
        \item Humans only have limited cognitive ability to deal with ``wicked'' systemic issues like climate change due to psychological distance, hyperbolic discounting, etc. \citep{schweizer2022SocialPerception}
        \vspace{0.5ex}
        \item This poses further research questions regarding Citizen's Councils' decision-making ability and effectiveness
    \end{itemize}
    
      
    \end{alertblock}
    \begin{block}{Limitations \& Next Steps}

    \begin{itemize}
        \item Data collection still ongoing 
        \item Likely smaller sample size than national surveys (limited population size \& non-response)
        \item Past participation could have induced treatment effects, potentially leading to sampling heterogeneity and confounding
    \end{itemize}
    \end{block}

    \vspace{-12mm}
    \begin{block}{Acknowledgments}
    Thanks to the \textit{nexus Institut} Berlin for supporting the data collection process and to Paul Sniderman for providing valuable feedback to earlier drafts. 
    
    \vspace{-12mm}
    \end{block}


    \begin{block}{References}
    \vspace{-3mm}
    \footnotesize
    \begin{flushleft}
        \bibliography{references}
    \end{flushleft}
    \end{block}
    
\end{column}

\separatorcolumn
\end{columns}
\end{frame}

\end{document}